\documentclass[UTF8]{article}

\usepackage{graphicx}
\usepackage{CJKutf8}
\usepackage{ctex}
\usepackage{cite}
\usepackage{amsmath,amssymb,amsfonts}
\usepackage{algorithmic}
\usepackage{textcomp}
\usepackage{xcolor}
\usepackage[font=small,]{caption}   %ʹͼ±í±êÌâ×ÖºÅСһºÅ
\def\abstractname{Summary}  %ÕªÒª±êÌâ

\begin{document}
\title{SAM}
\author{calabash\_boy}
\chapter{SAM}
\maketitle
\section{问题}
现在需要对字符串$S$建立一种数据结构,需要资瓷识别子串操作。即询问字符串$T$,判断$T$是否为$S$的子串,以及回答$T$在$S$中的出现次数,要求复杂度$O(|T|)$。
\subsection{Naive的想法}
首先有这样的一个事实:任何$S$的子串$S[l,r]$,都是$S$的前缀的后缀。具体而言就是$S[l,r]$是$S[1,r]$的后缀。
同时每个子串也都是原串的后缀的前缀。
\subsection{Naive的实现}
将原串的每一个后缀插入到$Trie$中,这样$Trie$中的每个点,代表的都是原串的后缀的前缀,于是可以实现识别每个子串。预处理复杂度$O(N^2)$.
\subsection{right集合定义:}
对于$S$的每一个子串$T$而言,$T$在$S$中的某一些位置出现,那么将$T$在$S$中出现位置的集合定义为\textbf{right 集合}。
\subsection{显然的观察1}
在\textbf{Naive的实现}中的$Trie$中,有很多状态可以合并,即观察到有一些状态代表的子串,其出现次数是相同的,更严格的讲:其每一个出现位置都是相同的。即\textbf{right 集合}相同。
\subsection{显然的观察2}
在\textbf{显然的观察1}中,提出了按照\textbf{right 集合}划分子串的等价类的思想。那么对于任意一个\textbf{right 集合},我们找到这个等价类中的最长的子串,记为$T_{longest}$。那么其他\textbf{right集合}相同的子串,一定都是$T_{longest}$的后缀。而且是连续的一些后缀,多么显然。。。
\subsection{后缀的符号表示}
使用$S_{T}(length)$表示串$T$的长度为$length$的后缀。
\subsection{显然的观察3}
令$T = S$,在\textbf{显然的观察2}中,我们知道$T$的一些长度最长的后缀的\textbf{right集合}和$T$是相同的,于是存在一个分界点,记为$l_p$,即$\forall i\in[l_p+1,n],S_{T}(i)$和$T$的\textbf{right集合}相同。而$\forall i\in[1,l_p],S_{T}(i)$的\textbf{right集合}都比$T$的要大,翻译一下就是说。。。更短的串,出现位置更多。。。
\subsection{后缀链接}\label{suffix_link}
至此,对于$T$,我们可以选取若干分点,记为$l_1 = 0,l_2,l_3,.....l_m = n$,使得$\forall i\in[l_x + 1,l_{x+1}],S_{T}(i)$都具有相同的\textbf{right 集合}。将他们用一条链链接起来,就形成了\textbf{后缀链接}。
\subsection{对Naive的实现进行优化}
由于我们有了很多\textbf{显然的观察},我们现在使用他们对\textbf{Naive的实现}进行优化,由于我们已经懂了\textbf{后缀链接}的意义,因此,我们只需要对$S$的每一个前缀,求解出其完整后缀链接,我们便得到了所有有用的状态。这也就是$SAM$所做的事情。同时也是\textbf{后缀树}所做的事情。。。(一下学了两个算法,好开心)

\vspace{\baselineskip}
\noindent\hrulefill

\section{后缀自动机}
\subsection{定义}
后缀自动机是一个自动机,它能够识别原串的每一个子串。翻译一下:他是一个$DAG$,每条边上有一个字符,每一个子串都可以在这个$DAG$上爬出来。其中$DAG$上的每个点与我们上面讲的状态对应。

\subsection{符号表示}
$nxt[x][c]$表示从$x$点,通过字符$c$可以到达的点。

$l[x]$表示$x$点所能表示的最长子串长度。

$fa[x]$表示$x$在后缀链接上的父亲。
\subsection{在线构造SAM}\label{construction}
假设我们已经构造了$S[1,x]$的$SAM$,即我们已经可以识别$S[1,x]$的所有子串,我们考虑如何修改得到$S[1,x+1]$的$SAM$,即向$SAM$中添加一个字符$c = S[x+1]$。

显然我们只考虑$S[1,x+1]$的所有后缀。也就是说我们要将$S[1,x+1]$的完整后缀链接放到$SAM$中去。

首先需要新建一个点$np$用来表示$S[1,x+1]$,他的\textbf{right集合}用已有$SAM$节点无法表示。记$last$表示$S[1,x]$所在的点。那么显然令$nxt[last][c] = np$,$l[np] = l[last] + 1$.

接下来我们要确定$fa[np]$。由于$S[1,x+1]$的每一个后缀,都是$s[1,x]$的后缀增加一个字符$c = S[x+1]$得到,因此我们其实只需要考察$S[1,x]$的\textbf{right集合}就可以得知$S[1,x+1]$的\textbf{right集合}如何变化。

沿着$last$的后缀链接走,如果一个点$p = fa[fa..[last]]$没有字符$c$的转移,那么增加$nxt[x][c] = np$.

直到我们找到了后缀链接上某点$p = fa[fa...[last]]$,他原本已经有字符$c$的转移,令$q = nxt[p][c]$,显然我们不需要再添加字符$c$的转移了,那么剩下的事情只剩下$fa[np]$等于啥。

由\textbf{后缀链接}的定义,如果$l[q] = l[p] + 1$,则$fa[np] = q$。

如果$l[q] > l[p] + 1$,you\textbf{后缀链接}定义,我们指导$fa[np]$节点的$l$ 一定等于$l[p] + 1$,因此我们将原节点$q$,分割为两部分:一部分代表长度$[l[p]+2,l[q]]$,另一部分代表长度$[l[fa[q]]+1,l[p]+ 1]$.

分割出的两个点,显然应该拥有相同的转移,即$nxt$相同。所以新建节点$nq$,令$nxt[nq] = nxt[q]$,$l[nq] = l[p] + 1$,这一步实现了分割长度。然后令$fa[nq] = fa[q]$,$fa[q] = nq$,这一步保证了后缀连接的正确性,这一步可以简单的类比于在链表中插入一个节点。之后我们就可以令$fa[np] = nq$了。之后修改$p$的剩下的后缀链接,使得原来通过字符$c$指向$q$的,都重新指向$nq$.
\subsection{复杂度证明}
简单的由\ref{suffix_link},复杂度并不能很清楚的证明。

但是由\ref{construction},我们很清楚的看到:复杂度为$O(|\Sigma|n)$

好像跟说了句:复杂度显然正确,没啥区别。。。

一切确实就是这么显然。。。

\vspace{\baselineskip}
\noindent\hrulefill

\section{应用}
回到我们的问题,我们要识别每个子串,很容易,在$SAM$上沿着字符边爬,能爬到就是子串,爬不到就不是子串。

emmmm。我们还需要统计子串出现次数。
\subsection{统计子串出现次数}
\textbf{SPOJ Substrings}

\textbf{https://www.spoj.com/problems/NSUBSTR}

由于后缀链接的定义,我们知道在$SAM$的一个节点中,它代表了一些连续长度的串,且其right集合相同。那么必然的,出现次数也相同,等于right集合大小。

如何求right集合大小呢,我们考虑每一个前缀$\forall x\in[1,n],S[1,x]$,我们找到所有包含$x$这个位置的节点,给他们的right集合大小+1即可。显然这些节点是前缀$S[1,x]$代表的整条后缀链接。

而由构造过程观察到所有点的后缀链接,其实形成的是一棵树。

那么问题变成了,树上$n$次修改,每次把根到一个点的路径都+1,仿佛闻到了傻逼题的味道?

做法为:开一个$cnt$数组,将$S$在$SAM$上运行一遍,将每个前缀所在点的$cnt = 1$.然后进行$\mbox{树上差分}dp$,或者\textbf{拓扑更新},whatever。
\subsection{统计本质不同子串个数}
\textbf{BZOJ 4516}

又闻到了傻逼题的味道?

$Ans = \sum{l[x] - l[fa[x]]}$
\subsection{后缀树的拓扑序}
我们发现在使用$SAM$的时候,经常需要在后缀树上进行\textbf{拓扑更新},而递归函数常数巨大,常规的拓扑排序……常数也不小。对于这样一个特殊的后缀树,我们是否可以快速处理出拓扑序,然后避免递归呢?

使用基数排序,以每个节点$x$的$l[x]$为关键字,排序之后即为拓扑序。因为沿着后缀树往根的方向走,$l[x]$单调变小。

\vspace{\baselineskip}
\noindent\hrulefill

\section{一些不厉害的题目}
\textbf{HDU 4641}

\textbf{POJ 1509}
\section{一些厉害的题目}
\textbf{codeforces gym 100962 D Deep Purple}

\textbf{SPOJ 7258}
\section{模板}
\textbf{https://github.com/4thcalabash/ACM-Code-Library} \\ \textbf{/blob/master/String/AutoMachine/Suffix\_Automaton.cpp}
\end{document}
